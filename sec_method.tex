% !TEX root = paper.tex
%\textbf{{[Analysis technique.]}}
\subsection{Experimental observables}
\label{sec:method}
%\subsubsection{cumulant analysis}
In this Letter we study the relationship between event-by-event fluctuations of magnitudes of two different flow harmonics of order $n$ and $m$ by using a recently proposed 4-particle observable~\cite{Bilandzic:2013kga}
%
\begin{eqnarray}
\left<\left<\cos(m\varphi_1\!+\!n\varphi_2\!-\!m\varphi_3-\!n\varphi_4)\right>\right>_c &=& \left<\left<\cos(m\varphi_1\!+\!n\varphi_2\!-\!m\varphi_3-\!n\varphi_4)\right>\right>\nonumber\\
&&{}-\left<\left<\cos[m(\varphi_1\!-\!\varphi_2)]\right>\right>\left<\left<\cos[n(\varphi_1\!-\!\varphi_2)]\right>\right>\nonumber\\
&=&\left<v_{m}^2v_{n}^2\right>-\left<v_{m}^2\right>\left<v_{n}^2\right>\,,%\nonumber\\
%&=&0\,.
\label{eq:4p_cumulant}
\end{eqnarray}
%
with the condition $m\neq n$ for two positive integers $m$ and $n$. We refer to these new observables as {\it Symmetric 2-harmonic 4-particle Cumulant}, and use notation SC$(m,n)$, or just SC. The double angular brackets indicate that the averaging procedure has been performed in two steps --- first over all distinct particle quadruplets in an event, and then in the second step the single-event averages were weighted with `number of combinations'. The latter for single-event average 4-particle correlations is mathematically equivalent to a unit weight for each individual quadruplet when the multiplicity differs event-by-event~\cite{Bilandzic:2012wva}. In both 2-particle correlators above all distinct particle pairs are considered in each case. The four-particle cumulant in Eq.~(\ref{eq:4p_cumulant}) is less sensitive to non-flow correlations than any 2- or 4-particle correlator on the right-hand side taken individually~\cite{Borghini:2001vi}. It is zero in the absence of flow fluctuations, or if the magnitudes of harmonics $v_m$ and $v_n$ are uncorrelated~\cite{Bilandzic:2013kga}. It is also unaffected by relationship between symmetry plane angles $\psi_m$ and $\psi_n$. The four-particle cumulant in Eq.~(\ref{eq:4p_cumulant}) is proportional to the linear correlation coefficient $c(a,b)$ introduced in~\cite{Niemi:2012aj} and discussed above, with $a=v_m^2$ and $b=v_n^2$. Experimentally it is more reliable to measure the higher order moments of flow harmonics $v_n^k\ (k \ge 2)$ with 2- and multiparticle correlation techniques~\cite{Borghini:2001vi,Bilandzic:2010jr,PhysRevC.44.1091}, than to measure the first moments $v_n$ with the event plane method, due to systematic uncertainties involved in the event-by-event estimation of symmetry planes~\cite{Poskanzer:1998yz,Luzum:2012da}. Therefore, we have used the new multiparticle observable in Eq.~(\ref{eq:4p_cumulant}) as meant to be the least biased measure of the correlation between event-by-event fluctuations of magnitudes of two different harmonics $v_m$ and $v_n$~\cite{Bilandzic:2013kga}.

The 2- and 4-particle correlations in Eq.~(\ref{eq:4p_cumulant}) were evaluated in terms of $Q$-vectors~\cite{Bilandzic:2010jr}. The $Q$-vector (or flow vector) in harmonic $n$ for a set of $M$ particles, where throughout this paper $M$ is multiplicity of an event, is defined as $Q_n\equiv\sum_{k=1}^Me^{in\varphi_k}$~\cite{Voloshin:1994mz,Barrette:1994xr}. We have used for a single-event average 2-particle correlation, $\left<\cos(n(\varphi_1\!-\!\varphi_2))\right>$, the following definition and analytic result in terms of $Q$-vectors:
%
\begin{equation}
\frac{1}{\binom{M}{2}2!}\,\sum_{\begin{subarray}{c}i,j=1\\ (i\neq j)\end{subarray}}^{M} e^{in(\varphi_i-\varphi_j)}
=\frac{1}{\binom{M}{2}2!}
\big[\left|Q_{n}\right|^2\!-\!M\big]\,.
\label{eq:two_n_n}
\end{equation}
%
For 4-particle correlation, $\left<\cos(m\varphi_1\!+\!n\varphi_2\!-\!m\varphi_3-\!n\varphi_4)\right>$, we used:
%
\begin{eqnarray}
&&\!\!\!\!\!\! \frac{1}{\binom{M}{4}4!}\,\sum_{\begin{subarray}{c}i,j,k,l=1\\ (i\neq j\neq k\neq l)\end{subarray}}^{M} e^{i(m\varphi_i+n\varphi_j-m\varphi_k-n\varphi_l)} = \nonumber\\
&&\!\!\!\!\!\!\frac{1}{\binom{M}{4}4!}\big[\left|Q_{m}\right|^2\left|Q_{n}\right|^2\!-\!
2\mathfrak{Re}\left[Q_{m+n}Q_{m}^*Q_{n}^*\right]\!-\!
2\mathfrak{Re}\left[Q_{m}Q_{m-n}^*Q_{n}^*\right]\nonumber\\
&&{}\!\!\!\!\!\!\!+\!\left|Q_{m+n}\right|^2\!+\!\left|Q_{m-n}\right|^2\!-\!(M\!-\!4)(\left|Q_{m}\right|^2\!+\!\left|Q_{n}\right|^2)
+\!M(M\!-\!6)\big]\,.
\label{eq:four_m_n_m_m}
\end{eqnarray}
%
\noindent In order to obtain the all-event average correlations, denoted by $\left<\left<\cdots\right>\right>$ in Eq.~(\ref{eq:4p_cumulant}), we have weighted single-event expressions in Eqs.~(\ref{eq:two_n_n}) and (\ref{eq:four_m_n_m_m}) with weights $M(M\!-\!1)$ and $M(M\!-\!1)(M\!-\!2)(M\!-\!3)$, respectively~\cite{Bilandzic:2012wva}.

$SC(m,n)$ can be normalized with the products  $\left<v_{m}^2\right>\left<v_{n}^2\right>$:
\begin{equation}
NSC(m,n) = SC(m,n) / \left<v_{m}^2\right>\left<v_{n}^2\right>.
\label{eq:nsc}
\end{equation}
This normalized $SC(m,n)$ ($NSC(m,n)$) reflects  the degree of the correlation which is expected to be insensitive to the magnitudes of $v_{m}$ and $v_{n}$, while $SC(m,n)$ contains both the degree of the correlation and individual $v_{n}$.
These products are obtained with two-particle correlations and using a psedorapidity gap of $|\Delta\eta|>1.0$ to suppress biases from few-particle non flow correlations.


