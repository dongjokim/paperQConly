The correlations between event-by-event fluctuations of amplitudes of anisotropic flow harmonics
in $\PbPb$ collisions at $\snn=2.76$~TeV have been measured with the ALICE detector at the Large Hadron Collider. 
The results were obtained with the new multi-particle cumulant method dubbed symmetric cumulants.
This method is robust against systematic biases originating from non-flow effects. 
The centrality dependence of correlation between the higher order harmonics ($v_3$, $v_4$, $v_5$) and the lower order harmonics ($v_2$, $v_3$) as well as the transverse momentum dependence of $v_3$-$v_2$ and $v_4$-$v_2$ correlations are presented. 
The results are compared to calculations from viscous hydrodynamics and  A Multi-Phase Transport ({AMPT}) models.
The comparisons to viscous hydrodynamic models demonstrate that
the different order Fourier harmonic correlations respond differently to the initial conditions or the shear viscosity to the entropy density ratio ($\eta/s$). The small $\eta/s$ regardless of initial conditions is favored and the small $\eta/s$ with the AMPT initial condition is closest to the results. 
$v_3$-$v_2$ and $v_4$-$v_2$ correlations show moderate $p_{\rm T}$ dependence in mid-central collisions. This might be an indication of possible viscous corrections for the equilibrium distribution at hadronic freeze-out, which might help to understand possible contribution of bulk viscosity in a hadronic phase of the system.
% which is the least understood part of hydrodynamic calculations.
%However the $v_3$-$v_2$ and  $v_4$-$v_3$ cannot be described by any model setting.
%pt dependence ??
Together with the existing measurements of individual flow harmonics the presented results provide further constraints 
on initial conditions and the transport properties of the system produced in heavy-ion collisions.
