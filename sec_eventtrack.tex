\subsection{Event and Track Selection}
\label{sec:experiment}
The data sample recorded by ALICE during the 2010 heavy-ion run at the
LHC is used for this analysis. Detailed descriptions of the ALICE
detector can be found
in~\cite{Aamodt:2008zz,Carminati:2004fp,Alessandro:2006yt}. The Time
Projection Chamber (TPC) was used to reconstruct charged particle
tracks and measure their momenta with full azimuthal coverage in the
pseudo-rapidity range $|\eta|<0.8$. Two scintillator
arrays (V0) which cover the pseudo-rapidity ranges $-3.7<\eta<-1.7$
and $2.8<\eta<5.1$ were used for triggering, and the determination of
centrality~\cite{Aamodt:2010cz}. The trigger
conditions and the event selection criteria are identical to those
described in~\cite{Aamodt:2010pa, Aamodt:2010cz}.
Approximately $10^7$ minimum-bias Pb-Pb events with
a reconstructed primary vertex within $\pm 10$ cm from the nominal
interaction point in the beam direction are used for this
analysis. Charged particles reconstructed in the TPC in $|\eta|<0.8$
and $0.2<\pt<5$ GeV/$c$ were selected. The charged track quality cuts
described in~\cite{Aamodt:2010pa} were applied to minimize
contamination from secondary charged particles and fake tracks. The
charged particle track reconstruction efficiency and contamination
were estimated from {\sc HIJING} Monte Carlo
simulations~\cite{Wang:1991hta} combined with a {\sc
GEANT3}~\cite{Brun:1994aa} detector model, and found to be independent of
the collision centrality. The reconstruction efficiency increases from
70\% to 80\% for particles with $0.2<\pt<1$~GeV/$c$ and remains
constant at $80 \pm 5$\% for $\pt>1$~GeV/$c$. The estimated
contamination by secondary charged particles from weak decays and
photon conversions is less than 6\% at $\pt=0.2$~GeV/$c$ and falls
below 1\% for $\pt>1$~GeV/$c$.
With this choice of low $\pt{}$ cut-off we are reducing event-by-event biases from smaller reconstruction efficiency 
at lower $\pt{}$, while the high $\pt{}$ cut-off was introduced to reduce the contribution to the anisotropies from jets. 
Reconstructed tracks were required to have at least 70 TPC space points (out of a maximum of 159). 
Only tracks with a transverse distance of closest approach (DCA) to the primary vertex less than 3 mm, both in longitudinal and transverse direction, are accepted to reduce the contamination from secondary tracks (for instance the charged particles produced 
in the detector material, particles from weak decays, etc.). 
Tracks with kinks (the tracks that appear to change direction due to multiple scattering, $K^{\pm}$ decays) were rejected.

