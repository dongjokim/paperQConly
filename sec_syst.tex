% !TEX root = paper.tex


\subsection{Systematic uncertainties}
\label{sec:uncertainties}
%\textbf{\tmp{[Track selection criteria.]}}

%p10================================================================================

%\textbf{\tmp{[An independent analysis with different tracks (if any).]}}

The systematic uncertainties are estimated by varying the event and track selection criteria. All systematic checks described here are performed independently. 
Each results of SC$(m,n)$ with a selected criterion are compared to ones from the default event and track selection described in the previous section.
The differences between the default results and the ones obtained from the variation of the selection criteria are taken as systematic uncertainty of each individual source.
The contributions from different sources were then added in quadrature to obtain the final value of the systematic uncertainty.

The event centrality was determined by the V0 detectors \cite{Abbas:2013taa} with better than 2\%  resolution of centrality determination. The systematic uncertainty from centrality determination was evaluated by using Time Projection Chamber (TPC) and Silicon Pixel Detector (SPD) \cite{Dellacasa:1999kf} detectors instead of the default, V0 detectors. The systematic uncertainties from the centrality determinations were about 3\% both for SC(5,2) and SC(4,3), 8\% for  SC(5,3).

%Because of limited acceptance of ALICE detector, primary vertex position along the longitudinal direction is important to ensure for the uniform pseudo-rapidity distribution over all events. 
As described in Sec.~\ref{sec:experiment}, the reconstructed vertex position in beam axis (z-vertex) is required to be located within 10 cm of interaction point (IP).  We measured the results with 8cm cut and the differences in the final results were less than 3\%.  

The analyzed events were recorded with two settings of the magnetic filed polarities and they have almost the same number of events. Both magnetic polarization events were used for the default analysis and the systematic uncertainties were evaluated from the results from each of two polarized magnetic field settings. 
Moreover, because of incompleteness of track reconstruction, correction steps are necessary to trace back from reconstructed tracks to the originally generated particles from the collisions. The effects from $p_T$ dependence reconstruction efficiency were taken into systematic uncertainty. Magnetic polarizations, reconstruction efficiency effects are relatively small and difference from the default settings were less than 2\%.

The systematic uncertainty due to the track reconstruction was estimated using two additional tracking, one relying on the so-called standalone TPC tracking with the 
same parameters described in Sec.~\ref{sec:experiment}, and the second that relies on the combination of the TPC and the ITS detectors with tighter selection criteria.
Since some parts of the SPD were switched off during some run periods, inefficient regions for common track reconstruction are apparent. To recover non-uniform azimuthal acceptance due to dead zones in SPD, and to get the best transverse momentum resolution, approach of hybrid selection with SPD hit and/or ITS refit tracks combine with TPC were used.  Then each track reconstruction was evaluated by varying the threshold on parameters used to select the tracks at the reconstruction level. 
The systematic difference up to 12\% was observed in SC$(m,n)$ results from the different track selections. 
In addition, we applied the like-sign technique to estimate non-flow correlation effects on SC$(m,n)$. The difference between all charged combinations and like-sign combinations were the largest contribution to the systematic uncertainty and they were  7\% for SC(4,3) and 20\% for SC(5,3). 

One of the other largest contributions to the systematic uncertainty originates from the non-uniform reconstruction efficiency. In order to estimate the effects on the measurements of these azimuthal correlators for various detector inefficiencies, we use the AMPT models (see the details in Sec.~\ref{sec:theory}) which have flat uniform distribution of azimuthal angles. Then we force to have detector inefficiencies by imposing non-uniform azimuthal distribution from the data. For the observables, SC(5,2), SC(5,3) and SC(4,3), the uncertainty from the non-uniform distribution of azimuthal angles were 9\%, 17\% and 11\%, respectively.
Generally, systematics uncertainties are large for the SC(5,3) and SC(5,2) than the lower harmonics of SC$(m,n)$, because smaller values of $v_n$ are sensitive to azimuthal modulation and $v_n$ decreases with $n$. Furthermore, the various AMPT models (see the details in Sec.~\ref{sec:theory}) were used to check the sensitivities of the observables due to the strength of the signals we are measuring, it turned out that the effect is negligible.
