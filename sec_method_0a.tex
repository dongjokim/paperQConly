% Experimental observables
%% Flow correlations in general
   % Niemi, ATLAS paper
%% Symmetric Cumulants (SC)
%% Normalized Symmetric Cumulants (NSC)
%% Summary of results of 1st ALICE paper on SC
%% Review of papers which cited ALICE 
% !TEX root = paper.tex

\section{Data Analysis}
\label{sec:method}
\subsection{Experimental observables}
%\subsubsection{cumulant analysis}

While from existing measurements an estimate can be placed on the average value of QGP's $\eta/s$, both at RHIC and LHC energies, what remains completely unknown is how the $\eta/s$ of QGP depends on temperature ($T$). This study has been just initiated by the theorists in Ref.~\cite{Niemi:2015qia}, where the first (and only rather qualitative) possibilities where investigated (see Fig.~1 therein). The emerging consensus of late is that it is unlikely that the study of individual flow harmonics $v_n$ will reveal the details of $\eta/s(T)$ dependence. In fact, in was demonstrated already in the initial study~\cite{Niemi:2015qia} that different $\eta/s(T)$ parameterizations can lead to the same centrality dependence of individual flow harmonics. In Ref.~\cite{Niemi:2012aj} new flow observables were introduced by the theorists, which quantify the degree of correlation between two different harmonics $v_m$ and $v_n$. The initial success of these new observables was attributed to their potential to discriminate for the first time the two respective contributions to anisotropic flow development---from initial conditions and from the transport properties of the QGP~\cite{Niemi:2012aj}. Therefore their mesurement in turn would enable the experimental verification of theoretical predictions for individual stages of heavy-ion evolution independently. Besides this advantage, it turned out that correlations of different flow harmonics are sensitive to the details of $\eta/s(T)$ dependence~\cite{ALICE:2016kpq}, to which individual flow harmonics are nearly insensitive~\cite{Niemi:2015qia}. 
 
 For technical reasons, discussed in detail in Refs.~\cite{ALICE:2016kpq,Bilandzic:2013kga}, the correlations between different flow harmonics cannot be studied experimentally with the same set of observables introduced by the theorists in Ref.~\cite{Niemi:2012aj}. Instead, in~\cite{Bilandzic:2013kga} the new flow observables obtained from multiparticle correlations, so-called \textit{Symmetric Cumulants~(SC)}, were introduced to quantify in the most realiable way (i.e. nearly insensitive to nonflow) the correlation of amplitudes of two different flow harmonics. The technical details are elaborated in Ref.~\cite{Bilandzic:2013kga}, while the first measurements of SC observables were recently released by ALICE Collaboration in Ref.~\cite{ALICE:2016kpq}.

SC($m$,$n$) can be normalized with the product $\left<v_{m}^2\right>\left<v_{n}^2\right>$ to obtain \textit{normalized} symmetric cumulants~\cite{ALICE:2016kpq,Giacalone:2016afq}, which we denote by NSC($m$,$n$), i.e.
%
\begin{equation}
\mathrm{NSC}(m,n) \equiv \frac{\mathrm{SC}(m,n)}{\left<v_{m}^2\right>\left<v_{n}^2\right>}\,.
\label{eq:nsc}
\end{equation}
%
Normalized symmetric cumulants reflect only the degree of the correlation which is expected to be insensitive to the magnitudes of $v_{m}$ and $v_{n}$, while SC$(m,n)$ contains both the degree of the correlation and individual $v_{n}$ harmonics. In Eq.~(\ref{eq:nsc}) the products in the denominator are obtained with two-particle correlations and using a psedorapidity gap of $|\Delta\eta|>1.0$ to suppress biases from few-particle nonflow correlations. On the other hand, in the two two-particle correlations which appear in the definition of SC$(m,n)$ the psedorapidity gap is not needed, since nonflow is suppressed by construction in SC observable, as the study based on HIJING model has clearly demontrated in Ref.~\cite{ALICE:2016kpq}.

The first measurements of SC observables have revealed that fluctuations of $v_2$ and $v_3$ are anti-correlated, while fluctuations of $v_2$ and $v_4$ are correlated in all centralities~\cite{ALICE:2016kpq}. However, the details of the centrality dependence differ in the fluctutation-dominated (most central) and the geometry-dominated (mid-central) regimes~\cite{ALICE:2016kpq}. Most importantly, the centrality dependence of SC(4,2) cannot be captured with the constant $\eta/s$ dependence, indicating clearly that the temperature plays an important role in describing QGP's $\eta/s$ dependence in various stages of heavy-ion evolution. These results were also used to discriminate between different parameterizations of initial conditions and it was demonstrated that in the fluctuation-dominated regime (in central collisions) MC-Glauber initial conditions with binary collisions weights are favored over wounded nucleon weights~\cite{ALICE:2016kpq}. 

\vspace{0.44cm}

\noindent\textbf{\textcolor{blue}{[Review of recent theoretical papers which discuss SC.]}} 
The SC observables provide orthogonal information to recently measured symmetry plane correlators in Refs.~\cite{ALICE:2011ab,Adare:2011tg,Aad:2014fla}. This statement does not exclude the possibility that both set of observables can be sensitive to the same physical mechanisms. In the recent theoretical study~\cite{Giacalone:2016afq} it was pointed out that the mechanism giving rise to symmetry plane correlations (nonlinear coupling) can also contribute to symmetric cumulants. As a concrete example it was discussed that the existing correlation due to hydrodynamic evolution between $V_4$ and $V_2^2$ (which are vectors in the transverse plane) implies that both the angles and the magnitudes are correlated~\cite{Giacalone:2016afq}. 

Interpretation of flow results obtained with multiparticle correlation techniques in small colliding systems, like pp and p--Pb at LHC, remains a challenge. The underlying difficulty stems from the fact that when anisotropic flow harmonic $v_n$ is estimated with $k$-particle correlator, the statistical spread of that estimate scales to leading order as $\sigma_{v_{n}}\sim\frac{1}{\sqrt{N}}\frac{1}{M^{k/2}}\frac{1}{v_{n}^{k-1}}$, where $M$ is the number of particles in an event (multiplicity) and $N$ is total number of events. This generic scaling ensures that multiparticle correlations are precision method only in heavy-ion collisions, characterized both with large values of multiplicity and flow. To leading order the measurements in small systems~\cite{Aad:2013fja,Abelev:2014mda,Khachatryan:2015waa,Adamczyk:2015xjc,Adare:2015ctn} and the measurements in heavy-ion collisions resemble the same features, which can be attributed to collective anisotropic flow in both cases. However, such interpretation is challenged by the outcome of recent Monte Carlo study~\cite{Loizides:2016tew} for $e^+e^-$ systems in which collective effects are not expected. Nonetheless, in this study to leading order multiparticle correlations exhibit yet again the similar universal trends first seen in heavy-ion collisions, both for elliptic and triangular flow. Therefore, it seems unlikely that the analysis of individual flow harmonics with multiparticle techniques will answer whether collective effects can develop and QGP be formed in small systems---instead new observables, like SC, might provide the final answer due to their better sensitivity~\cite{Niemi:2012aj,ALICE:2016kpq}.

